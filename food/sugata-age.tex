\documentclass{minimal}

\usepackage{xeCJK}
\usepackage{ruby}
\usepackage{fontsize}
\usepackage{xcolor}
\usepackage{fontspec}

\changefontsize[55pt]{24pt}
%\setCJKmainfont{MS Gothic}
\setCJKmainfont{UD Digi Kyokasho NK-R}

\newCJKfontfamily\FontR{UD Digi Kyokasho NK-R}
\newCJKfontfamily\FontB{UD Digi Kyokasho NK-B}

\renewcommand{\rubysep}{-1.5ex}
\renewcommand{\rubysize}{0.5}

\begin{document}

\noindent
{\FontB\ruby{姿}{すがた}揚げ}
\vspace{24pt}

\noindent
\ruby{魚介類}{ぎょかいるい}や肉類、
野菜類などを\ruby{油}{あぶら}で揚げた揚げ物の\ruby{一種}{いっしゅ}で、
\ruby{食材}{しょくざい}の\ruby{元}{もと}の形に\ruby{近}{ちか}い\ruby{形状}{けいじょう}で揚げたもの。
またはその\ruby{調理法}{ちょうりほう}。

\vspace{24pt}
\noindent
\ruby{主}{おも}に魚の調理法として\ruby{用}{もち}いられることが多く、
\ruby{隠}{かく}し\ruby{包丁}{ぼうちょう}を入れて揚がりやすくすることも多い。


\end{document}
