\documentclass{minimal}
\usepackage[letterpaper,left=3cm,right=3cm,
    top=3cm,bottom=2cm,includefoot,footskip=2cm]{geometry}
\usepackage{fancyhdr} 

\usepackage{xeCJK}
\usepackage{ruby}
\usepackage{fontsize}
\usepackage{xcolor}
\usepackage{fontspec}

\changefontsize[40pt]{20pt}
\setCJKmainfont{UD Digi Kyokasho NK-R}

\renewcommand{\rubysep}{-1.3ex}
\renewcommand{\rubysize}{0.5}

\setlength{\parindent}{0pt}
\setlength{\parskip}{10pt}

\pagestyle{fancy} 

\newfontfamily\Footer{Tahoma}

\fancyhf{}
\fancyfoot[C]{\Footer{\fontsize{12}{14}\selectfont \thepage}}
\renewcommand{\headrulewidth}{0pt}

% -----------------------------------------------------------------------------

\begin{document}

とても\ruby{少}{すく}ない

ほとんどいない

\ruby{当時}{とうじ}、\ruby{先}{さき}のことは\ruby{誰}{だれ}にもわかりませんでした

\ruby{青豆}{あおまめ}

\ruby{枝豆}{えだまめ}さん

\ruby{空豆}{そらまめ}さん

いいえ、枝豆(空豆)ではなく、青豆です。
まあ\ruby{似}{に}たようなものですが

いや、それにしても\ruby{珍}{めずら}しいお名前ですね

良い車ですね。とても\ruby{静}{しず}かだし

なんていう車なんですか?

トヨタのクラウン?ロイヤルサルーン

音楽がきれいに聞こえる

静かな車です。それもあってこの車を選んだんです。こと遮音にかけてはトヨタは世界でも有数の技術を持っていますから

たしかに静か

それにステレオの装置もずいぶん高級なものみたい

買うときには、決断が必要でした

でもこのように長い時間を車内で過ごしますから、できるだけ良い音を聴いていたいですし、また——

ヤナーチェック

なんですか?

ヤナーチェック。この音楽を作曲した人

知りませんね

チェコの作曲家

ほう

これは個人タクシーですか?

そうです

個人でやってます。この車は二台目になります

シートの座り心地がとてもいい

ありがとうございます。ところでお客さん

ひょっとしてお急ぎですか?

渋谷で人と待ち合わせがあります。だから首都高に乗ってもらったんだけど

何時に待ち合わせてます?

四時半

今が三時四十五分ですね。これじゃ間に合わないな

そんなに渋滞はひどいの?

前の方でどうやらでかい事故があったようです。普通の渋滞じゃありません。さっきからほとんど前に進んでいませんから

交通情報を聞かなくても、そういうことはわかるの?

交通情報なんてあてになりゃしません

あんなもの、半分くらいは嘘です。道路公団が自分に都合のいい情報を流しているだけです。今ここで本当に何が起こっているかは、自分の目で見て、自分の頭で判断するしかありません

それであなたの判断によれば、この渋滞は簡単には解決しない?

当分は無理ですね

そいつは保証できます。いったんこうがちがちになっちまうと、首都高は地獄です。待ち合わせは大事な用件ですか?

ええ、とても。クライアントとの待ち合わせだから

そいつは困りましたね。お気の毒ですが、たぶん間に合いません

じゃあ、どうすればいいのかしら?

どうしようもありません。ここは首都高速道路ですから、次の出口にたどり着くまでは手の打ちようがないです。一般道路のようにちょっとここで降りて、最寄りの駅から電車に乗るというわけにはいきません

次の出口は?

池尻ですが、そこに着くには日暮れまでかかるかもしれませんよ

高速道路では時間料金は加算されません

だから料金のことは心配しなくていいです。でもお客さん、待ち合わせに遅れると困るでしょう?

もちろん困るけど、でも手の打ちようもないんでしょう?

あのですね、方法がまったくないってわけじゃないんです。いささか強引な非常手段になりますが、ここから電車で渋谷まで行くことはできます

非常手段?

あまりおおっぴらには言えない方法ですが

ほら、あの先に車を寄せるスペースがあるでしょう

エッソの大きな看板が立っているあたりです

実はですね、地上に降りるための階段があそこにあります。火災とか大地震が起きたときに、ドライバーが車を捨ててそこから地上に降りられるようになっているわけです。普段は道路補修の作業員なんかが使っています。その階段を使って下に降りれば、近くに東急線の駅があります。そいつに乗れば、あっという間に渋谷です

首都高に非常階段があるなんて知らなかった

一般にはほとんど知られてはいません

しかし緊急事態でもないのに、その階段を勝手に使ったりすると、問題になるんじゃないかしら?

どうでしょうね。道路公団の細かい規則がどうなっているのか、私にもよくわかりません。しかし誰に迷惑をかけることでもなし、大目に見てもらえるのではないでしょうか。だいたいそんなところ、誰もいちいち見張っちゃいません。道路公団ってのはどこでも職員の数こそ多いけど、実際に働いている人間が少ないことで有名なんです

どんな階段?

そうですね、火災用の非常階段に似ています。ほら、古いビルの裏側によくついているようなやつ。とくに危険はありません。高さはビルの三階ぶんくらいありますが、普通に降りられます。いちおう入り口のところに柵がついていますが、高いものじゃないし、その気になればわけなく乗り越えられます

運転手さんはその階段を使ったことがあるの?

あくまでお客さん次第です

ここに座って良い音で音楽を聴きながら、のんびりしてらしても、私としちゃちっともかまいません。いくらがんばってもどこにも行けないんですから、こうなったらお互い腹をくくるしかありません。しかしもし緊急の用件がおありなら、そういう非常手段も<傍点>なくはない</傍点>ってことです

ここで降ります。遅れるわけにはいかないから

領収書は?

けっこうです。お釣りもいらない

それはどうも

風が強そうですから、気をつけて下さい。足を滑らせたりしないように

気をつけます

それから

ひとつ覚えておいていただきたいのですが、ものごとは見かけと違います

それはどういうことかしら?

つまりですね、言うなればこれから<傍点>普通ではない</傍点>ことをなさるわけです。そうですよね? 真っ昼間に首都高速道路の非常用階段を降りるなんて、普通の人はまずやりません。とくに女性はそんなことしません

そうでしょうね

で、そういうことをしますと、そのあとの日常の風景が、なんていうか、いつもとはちっとばかし違って見えてくるかもしれない。私にもそういう経験はあります。でも見かけにだまされないように。現実というのは常にひとつきりです

現実はいつだってひとつしかありません

もちろん

とても良い音だった

作曲家の名前はなんて言いましたっけ?

ヤナーチェック

ヤナーチェック

お気をつけて。約束の時間に間に合うといいんですが

ねえねえ、あの女の人、何しているの?どこにいくの?

私も外に出て歩きたい。ねえ、お母さん、私も外に出たい。ねえ、お母さん

発作

よう、天吾くん

どうした。また例のやつか? 大丈夫か?

すみません。もう大丈夫です

それって、何かの発作じゃないよな?

たいしたことじゃありません。ただの立ちくらみのようなものです。ただきついだけで

車を運転してるときなんかにそういうのがおこると、なかなか大変そうだ

車の運転はしません

それはなによりだ。知り合いにスギ花粉症の男がいてね、運転中にくしゃみが始まって、そのまま電柱にぶつかっちまった。ところが天吾くんのは、くしゃみどころじゃすまないものな。最初のときはびつくりしたよ。二回目ともなれば、まあ少しは慣れてくるけど

すみません

新しい水をもらおうか?

いえ、大丈夫です。もう落ち着きました

それで、何の話をしていたんでしたっけ?

ええと、俺たち何を話してたんだっけな


\end{document}
